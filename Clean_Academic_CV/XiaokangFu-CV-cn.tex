% ========================================================
% 付小康简历 - 中文版 (合并版)
% 基于 Matthew DeVerna 的模板 (www.matthewdeverna.com)
% ========================================================

\documentclass[11pt]{article}
\usepackage[margin=1in]{geometry}

% Citation style
\usepackage[
    backend=biber,
    maxnames=20,
    style=nature,
    sorting=ydnt,
    defernumbers=true,
]{biblatex}
\DeclareFieldFormat[article]{volume}{#1}
\addbibresource{ref.bib}

% Packages
\usepackage{longtable}
\usepackage{bookmark}
\usepackage{fontawesome5}
\usepackage{ragged2e}
\usepackage{soul}
\usepackage{fontspec}
\setmainfont{Times New Roman}
\usepackage{xeCJK}
\setCJKmainfont{STSong}
\usepackage{xcolor}
\usepackage{hyperref}

% Custom colors
\definecolor{firebrick}{HTML}{b22222}
\definecolor{darkslategrey}{HTML}{2f4f4f}
\definecolor{cornflowerblue}{HTML}{6495ed}
\definecolor{mediumslateblue}{HTML}{7b68ee}

\hypersetup{
    colorlinks=true,
    breaklinks=true,
    linkcolor=cornflowerblue,
    urlcolor=cornflowerblue,
    anchorcolor=cornflowerblue,
    citecolor=cornflowerblue,
    pdftitle={付小康简历},
    pdfauthor={付小康},
    bookmarksopen=true,
}

\newcommand{\mytitle}[4]{
  \begin{center}
    \Large\textbf{#1}\normalsize \\
    \href{mailto:#2}{#2} \\
    \href{https://#3}{#3} \\
    #4
  \end{center}
}

\begin{document}

% ==================== Header ====================
\definecolor{emailcolor}{HTML}{666666}
\definecolor{webcolor}{HTML}{666666}
\definecolor{githubcolor}{HTML}{333333}
\definecolor{starcolor}{HTML}{F5A623}
\definecolor{googlecolor}{HTML}{4285F4}
\begin{center}
    \LARGE\textbf{付小康}\normalsize \\[0.4em]
    {\color{emailcolor}\faIcon{envelope}}~\href{mailto:fxk123@gmail.com}{fxk123@gmail.com} \quad
    {\color{webcolor}\faIcon{globe}}~\href{https://wybert.github.io}{wybert.github.io} \\[0.3em]
    {\color{githubcolor}\faIcon{github}}~\href{https://github.com/wybert}{GitHub} \textbf{250} {\color{starcolor}\faIcon{star}} \quad
    {\color{googlecolor}\faIcon{google}}~\href{https://scholar.google.com.hk/citations?user=H9RAbHIAAAAJ}{Google Scholar} \textbf{1020} citations
\end{center}
\vspace{0.5em}

\RaggedRight

%% ==================== 研究兴趣 ====================
\section*{研究兴趣}

我的研究致力于通过创新技术和跨学科合作, 推动可持续发展. 我的研究兴趣包括地理信息科学 (GIS), 地理空间人工智能 (GeoAI), 环境管理, 灾害响应, 公共卫生, 城市计算, 社会行为分析, 以及开发可再现和可扩展的计算工具.

%% ==================== 教育经历 ====================
\pdfbookmark[1]{Education}{}
\section*{教育经历}

\renewcommand{\thefootnote}{\fnsymbol{footnote}}
\setcounter{footnote}{0}

\begin{longtable}[l]{@{}p{.125\textwidth} p{0.875\textwidth}}
    2015-2020 & \textbf{博士}, 制图学与地理信息工程, 武汉大学 \\
    2013-2015 & \textbf{硕士}, 测绘工程, 武汉大学 \\
    2009-2013 & \textbf{学士}, 测绘工程, 内蒙古科技大学
\end{longtable}

\footnotetext[1]{Expected.}
\renewcommand{\thefootnote}{\arabic{footnote}}
\setcounter{footnote}{1}

%% ==================== 工作经历 ====================
\pdfbookmark[1]{Research Experience}{exp_research}
\section*{工作经历}
\label{exp_research}

\begin{longtable}[l]{@{}p{.125\textwidth} p{0.875\textwidth}}

2023-现在 & 博士后, \href{https://gis.harvard.edu/}{地理分析中心}, 哈佛大学, (合作导师: \href{https://gis.harvard.edu/people/s-v-subramanian}{S. V. Subramanian博士}) \\

2021-2023 & 访问研究员, \href{https://gis.harvard.edu/}{地理分析中心}, 哈佛大学, (合作导师: \href{https://gis.harvard.edu/people/s-v-subramanian}{S. V. Subramanian博士}) \\

2020-2023 & 博士后, \href{http://www.lmars.whu.edu.cn/en/}{测绘遥感信息工程国家重点实验室}, 武汉大学, (合作导师: \href{https://en.whu.edu.cn/info/1073/1568.htm}{龚健雅博士}) \\

\end{longtable}

%% ==================== 发表记录 ====================
\pdfbookmark[1]{Publications}{pubs}
\section*{发表记录}
\label{pubs}

\vspace{-.75em}
\small
\faIcon{google}~\href{https://scholar.google.com.hk/citations?user=H9RAbHIAAAAJ}{Google Scholar}\\
$\dagger \rightarrow$ Equal contribution
\normalsize

\subsection*{筛选论文}
\label{select-paper}
\newrefcontext[labelprefix=S]
\nocite{*}
\printbibliography[type=article, heading=none, resetnumbers=true, keyword=S]

\pdfbookmark[2]{Journal Articles}{journal-article}
\subsection*{其他论文}
\label{journal-article}
\newrefcontext[labelprefix=J]
\nocite{*}
\printbibliography[type=article, heading=none, resetnumbers=true, keyword=J]

\pdfbookmark[2]{Working papers}{working-papers}
\subsection*{待发表论文}
\label{working-papers}
\newrefcontext[labelprefix=W]
\printbibliography[type=misc,heading=none,resetnumbers=true,keyword=R]

%% ==================== 主持的项目 ====================
\section*{主持的项目}

\begin{enumerate}
    \item 2024. Host: Developing Workbenches for Spatial Data Science.The I/UCRC for Spatiotemporal Thinking, Computing and Applications (NSF USA).
    \item 2021. Host: Regional Health Index Calculation Model Based on Multi-Source Big Data. State Key Laboratory of Information Engineering in Surveying, Mapping and Remote Sensing.
\end{enumerate}

%% ==================== 开发的软件和工具 ====================
\pdfbookmark[1]{Tools \& Software}{tools}
\section*{开发的软件和工具}
\label{tools}

\subsection*{\textbf{开源可再现与可复制的地理信息科学}}
\begin{itemize}
    \item[] \textbf{Geospatial Analytics Extension for KNIME}: 用于可再现与可复制地理信息科学的可视化编程工具,包括地理空间数据的读取, 写入, 处理, 分析, 建模和可视化.(\href{https://github.com/spatial-data-lab/knime-geospatial-extension}{GitHub} | \href{https://sdl.gis.harvard.edu/sites/projects.iq.harvard.edu/files/agenda_2024_3_1.pdf}{2024时空数据科学研讨会报告})

    \item[] \textbf{Dataverse Extension for KNIME}: 与Dataverse平台交互的可视化编程工具, 包括上传至Dataverse, 从Dataverse下载, 搜索和读取数据.(\href{https://github.com/spatial-data-lab/knime-dataverse-extension}{GitHub} | \href{https://projects.iq.harvard.edu/dcm2022/agenda}{2022 Dataverse社区会议报告}和\href{https://gis.harvard.edu/event/integrating-cloud-computing-knime-platform-monitoring-deforestation}{2023哈佛研讨会})

    \item[] \textbf{Google Earth Engine Extension for KNIME}: 与Google Earth Engine交互的可视化编程工具.(\href{https://github.com/spatial-data-lab/knime-gee-extension}{GitHub} | \href{https://gis.harvard.edu/event/integrating-cloud-computing-knime-platform-monitoring-deforestation}{2024 ABCD-GIS/地理学系列讲座报告})
\end{itemize}

\subsection*{开源地理空间数据科学}

\begin{itemize}
    \item[] \textbf{Georouting}: 为Python用户提供的路径计算工具, 支持OSRM, 谷歌地图, 必应地图等多种路径规划工具, 并提供统一的API接口.(\href{https://pypi.org/project/georouting/}{PyPi} | \href{https://github.com/wybert/georouting}{GitHub} | \href{https://www.aag.org/wp-content/uploads/2023/03/AAG-2023-PDF-program.pdf}{2023年美国地理学年会提及})

    \item[] \textbf{Geopandas}: 面向地理数据处理的Python工具.(贡献1个\href{https://github.com/geopandas/geopandas/pull/2658}{PR} | \href{https://github.com/geopandas/geopandas}{GitHub})
\end{itemize}

\subsection*{地理空间大数据}

\begin{itemize}
    \item[] \textbf{\href{https://gis.harvard.edu/rapidroute}{RapidRoute}}: 开源的旅行时间估算系统.

    \item[] \textbf{\href{https://gis.harvard.edu/billion-object-platform-v20}{Billion Object Platform}}: 实时处理数十亿条记录的地理空间分析平台.(\href{https://www.iq.harvard.edu/news/cga-geospatial-infrastructure-enhancement}{2024 IQSS新闻通讯报道})
\end{itemize}

%% ==================== 学术交流和会议 ====================
\pdfbookmark[1]{Presentations}{presentations}
\section*{学术交流和会议}
\label{presentations}

\vspace{-.75em}
\small
$\dagger \rightarrow$ Equal contribution
\normalsize

\pdfbookmark[2]{Talks}{talks}
\subsection*{口头报告}
\label{talks}
\newrefcontext[labelprefix=T]
\printbibliography[type=misc,heading=none,resetnumbers=true,keyword=T]

\pdfbookmark[2]{Posters}{posters}
\subsection*{海报}
\label{posters}
\newrefcontext[labelprefix=P]
\printbibliography[type=misc,heading=none,resetnumbers=true,keyword=P]

%% ==================== 教学 ====================
\pdfbookmark[1]{Teaching}{teaching}
\section*{教学}
\label{teaching}

\subsection*{哈佛大学}

\begin{longtable}[l]{@{}p{.125\textwidth} p{0.875\textwidth}}

    2024 & 讲师及组织者, 研讨会: 面向时空数据科学的KNIME Business Hub.(\href{https://gis.harvard.edu/event/workshop-web-based-workflows-reproducible-and-shareable-spatiotemporal-data-analysis}{Link})   \\
    2023 & 讲师, 哈佛大学时空数据科学暑期工作坊.(\href{https://sdl.gis.harvard.edu/event/summer-workshop-spatiotemporal-innovation-0}{Link}) \\
    2023 & 报告人, 研讨会: 无代码可视化编程的时空数据分析.(\href{https://gis.harvard.edu/event/spatiotemporal-data-analysis-codeless-visual-programming}{Link}) \\

\end{longtable}

%% ==================== 学术服务 ====================
\pdfbookmark[1]{Academic Service}{service}
\section*{学术服务}
\label{service}

\subsection*{委员和会员}

\begin{longtable}[l]{@{}p{.125\textwidth} p{0.875\textwidth}}

 2023-至今 & 地理信息科学大学联盟(UCGIS)传播委员会委员 \\
    2023-至今 & 美国地理学家协会(AAG)会员 \\
    2024 & 电气与电子工程师协会(IEEE)会员 \\

\end{longtable}

\subsection*{Journal Reviewer}

\begin{longtable}[l]{@{}p{.125\textwidth} p{0.875\textwidth}}

    2025 & International Journal of Applied Earth Observation and Geoinformation, Ecological Indicators\\
    2024 & Land, Sustainability, International Journal of Applied Earth Observation and Geoinformation \\

\end{longtable}

%% ==================== Footer ====================
\centering
\rule{0.25\linewidth}{0.4pt}\\
\medskip
最近更新: \today

\end{document}
