
My research is dedicated to advancing sustainable development through innovative technologies and interdisciplinary collaboration. My interests including geographic information science (GIS), geospatial artificial intelligence (GeoAI), environmental management, disaster response, public health, urban computing and sustainability, social behavior analysis, digital twins, and the development of reproducible and scalable computational tools. 

% 从您的研究来看,可以从以下应用领域进行划分:

% 1. 环境监测与可持续发展(Environmental Monitoring and Sustainability)
% 	•	研究重点:
% 	•	通过社交媒体和空间数据分析监测空气质量、降水模式等环境现象。
% 	•	预测和缓解环境风险,例如洪水管理和气候变化影响评估。
% 	•	相关论文:
% 	•	“Using social media to detect outdoor air pollution and monitor air quality index (AQI)”
% 	•	“Precipitation projections using a spatiotemporally distributed method”
% 	•	“Inferring urban air quality based on social media”
% 	•	“Modeling urban air quality trend surface using social media data”
% 	•	应用场景:
% 	•	发展实时环境监控工具,为政策制定提供科学支持,优化城市可持续发展规划。

% 2. 城市韧性与灾害管理(Urban Resilience and Disaster Management)
% 	•	研究重点:
% 	•	通过分析社交媒体数据和其他空间数据识别灾害的发生、演化及影响。
% 	•	为城市洪水管理和灾害应急响应提供决策支持。
% 	•	相关论文:
% 	•	“Community evolutional network for situation awareness using social media”
% 	•	“A new method to detect the development situation of disasters based on social media co-word network”
% 	•	“Stream flow simulation and verification in ungauged zones by coupling hydrological and hydrodynamic models”
% 	•	应用场景:
% 	•	提升城市抗灾能力,发展预测与实时响应系统。

% 3. 医疗健康与公共卫生(Healthcare and Public Health)
% 	•	研究重点:
% 	•	通过构建可扩展模型评估全球医疗资源分布与可达性。
% 	•	分析社交媒体中的心理健康信号和疾病传播模式。
% 	•	相关论文:
% 	•	“Spatiotemporal characteristics and factor analysis of SARS-CoV-2 infections among healthcare workers in Wuhan, China”
% 	•	“Human mobility and COVID-19 transmission: a systematic review”
% 	•	“Public surveillance of social media for suicide using advanced deep learning models”
% 	•	“Global Healthcare Accessibility and Drive Time Estimation on Geospatial Big Data”
% 	•	应用场景:
% 	•	为优化医疗资源配置、健康干预措施和流行病预测提供数据支持。

% 4. 人类流动性与社会经济分析(Human Mobility and Socioeconomic Analysis)
% 	•	研究重点:
% 	•	基于人类流动性模式的空间分析研究城市经济分隔和社会不平等问题。
% 	•	探索人类流动性与城市规划、经济活动之间的相互作用。
% 	•	相关论文:
% 	•	“Human mobility data in the COVID-19 pandemic: characteristics, applications, and challenges”
% 	•	“A realistic and multilevel measurement of citywide spatial patterns of economic segregation based on human activities”
% 	•	“Assessing reliability of Chinese geotagged social media data for spatiotemporal representation of human mobility”
% 	•	应用场景:
% 	•	促进公平的城市规划政策制定,评估交通和基础设施建设对流动性和经济活动的影响。

% 5. 社会心理与行为分析(Social Psychology and Behavioral Analysis)
% 	•	研究重点:
% 	•	使用社交媒体数据分析多维情绪和心理健康信号。
% 	•	研究广告中的种族和性别歧视模式。
% 	•	相关论文:
% 	•	“Revealing the spatial co-occurrence patterns of multi-emotions from social media data”
% 	•	“Social media space provides public surveillance for suicide”
% 	•	“Racial discrimination patterns in advertising using social media data”
% 	•	应用场景:
% 	•	支持心理健康监测和社会行为分析,减少社会偏见。

% 6. 大规模数据分析与工具开发(Big Data Analysis and Tool Development)
% 	•	研究重点:
% 	•	开发高性能、可复现的地理空间数据处理工具,降低技术壁垒。
% 	•	应用GPU和高性能计算进行大规模数据分析与可视化。
% 	•	相关论文:
% 	•	“Geospatial Analytics Extension for KNIME”
% 	•	“Elevating the RRE Framework for Geospatial Analysis with Visual Programming Platforms”
% 	•	“A comparative study of methods for drive time estimation on geospatial big data”
% 	•	应用场景:
% 	•	提供通用平台,支持不同领域研究者高效处理大规模数据。

% 总结:研究应用领域的聚焦

% 您的研究以数据驱动的决策支持为核心,广泛覆盖环境管理、灾害应急、公共卫生、社会行为分析和大规模计算工具开发。您的工作致力于通过创新技术和跨学科合作推动社会公平与可持续发展。