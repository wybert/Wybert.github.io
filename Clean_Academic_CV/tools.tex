
\subsection*{\textbf{Open Source Reproducible and Replicable GIScience}}
\begin{itemize}
    \item[] \textbf{Geospatial Analytics Extension for KNIME}: Visual Programming Tools for Reproducible and Replicable GIScience, including read, write, process, analysis, modeling, and visualization geospatial data. (\href{https://github.com/spatial-data-lab/knime-geospatial-extension}{GitHub} | \href{https://sdl.gis.harvard.edu/sites/projects.iq.harvard.edu/files/agenda_2024_3_1.pdf}{Talk in The Symposium on Spatiotemporal Data Science, 2024}
    
    \item[] \textbf{Dataverse Extension for KNIME}: Visual Programming Tools for interacting with Dataverse platform, including upload, download from or to Dataverse; search, read data from Dataverse. (\href{https://github.com/spatial-data-lab/knime-dataverse-extension}{GitHub} | \href{https://projects.iq.harvard.edu/dcm2022/agenda}{Talk in Dataverse Community Meeting, 2022} and \href{https://gis.harvard.edu/event/integrating-cloud-computing-knime-platform-monitoring-deforestation}{Seminar in Harvard, 2023})

    \item[] \textbf{Google Earth Engine Extension for KNIME}: Visual Programming Tools for interacting Google Earth Engine Extension. (\href{https://github.com/spatial-data-lab/knime-gee-extension}{GitHub} | \href{https://gis.harvard.edu/event/integrating-cloud-computing-knime-platform-monitoring-deforestation}{Talk in ABCD-GIS / Geography Colloquium, 2024})
    
    
\end{itemize}


\subsection*{Open Source Geospatial Data Science}

\begin{itemize}
    \item[] \textbf{Georouting}. Routing calculation for Python users, supporting most of the routing tools, including OSRM, Google Maps, Bing Maps, etc. with a unified API. (\href{https://pypi.org/project/georouting/}{PyPi} | \href{https://github.com/wybert/georouting}{GitHub} | \href{https://www.aag.org/wp-content/uploads/2023/03/AAG-2023-PDF-program.pdf}{Mentioned in AAG, 2023})

    \item[] \textbf{Geopandas}. Python tools for geographic data. (Contribute 1  \href{https://github.com/geopandas/geopandas/pull/2658}{Pr} | \href{https://github.com/geopandas/geopandas}{GitHub})
    
\end{itemize}

\subsection*{Geospatial Big Data}

\begin{itemize}
    \item[] \textbf{\href{https://gis.harvard.edu/rapidroute}{RapidRoute}}. Rapid Route is an open-source system developed for estimating travel times. 

    % \item[] \textbf{\href{https://dataverse.harvard.edu/dataset.xhtml?persistentId=doi:10.7910/DVN/IAYJOC}{Dataset: Harvard CGA Geotweet Census Archive}}. A subset of Harvard CGA Geotweet Archive v2.0 enriched with nationwide census data. It contains the tweet and user identification records along with census variables for more than 2 billion geo-tagged tweets from January 2012 to July 2023.

    \item[] \textbf{\href{https://gis.harvard.edu/billion-object-platform-v20}{Billion Object Platform}}. Real-time geospatial analsyis with billions records. (\href{https://www.iq.harvard.edu/news/cga-geospatial-infrastructure-enhancement}{Report in IQSS News Letter, 2024})
    
\end{itemize}