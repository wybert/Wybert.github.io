\subsection*{\textbf{开源可再现与可复制的地理信息科学}}
\begin{itemize}
    \item[] \textbf{Geospatial Analytics Extension for KNIME}: 用于可再现与可复制地理信息科学的可视化编程工具,包括地理空间数据的读取, 写入, 处理, 分析, 建模和可视化.(\href{https://github.com/spatial-data-lab/knime-geospatial-extension}{GitHub} | \href{https://sdl.gis.harvard.edu/sites/projects.iq.harvard.edu/files/agenda_2024_3_1.pdf}{2024时空数据科学研讨会报告})
    
    \item[] \textbf{Dataverse Extension for KNIME}: 与Dataverse平台交互的可视化编程工具, 包括上传至Dataverse, 从Dataverse下载, 搜索和读取数据.(\href{https://github.com/spatial-data-lab/knime-dataverse-extension}{GitHub} | \href{https://projects.iq.harvard.edu/dcm2022/agenda}{2022 Dataverse社区会议报告}和\href{https://gis.harvard.edu/event/integrating-cloud-computing-knime-platform-monitoring-deforestation}{2023哈佛研讨会})

    \item[] \textbf{Google Earth Engine Extension for KNIME}: 与Google Earth Engine交互的可视化编程工具.(\href{https://github.com/spatial-data-lab/knime-gee-extension}{GitHub} | \href{https://gis.harvard.edu/event/integrating-cloud-computing-knime-platform-monitoring-deforestation}{2024 ABCD-GIS/地理学系列讲座报告})
\end{itemize}


\subsection*{开源地理空间数据科学}

\begin{itemize}
    \item[] \textbf{Georouting}: 为Python用户提供的路径计算工具, 支持OSRM, 谷歌地图, 必应地图等多种路径规划工具, 并提供统一的API接口.(\href{https://pypi.org/project/georouting/}{PyPi} | \href{https://github.com/wybert/georouting}{GitHub} | \href{https://www.aag.org/wp-content/uploads/2023/03/AAG-2023-PDF-program.pdf}{2023年美国地理学年会提及})

    \item[] \textbf{Geopandas}: 面向地理数据处理的Python工具.(贡献1个\href{https://github.com/geopandas/geopandas/pull/2658}{PR} | \href{https://github.com/geopandas/geopandas}{GitHub})
\end{itemize}

\subsection*{地理空间大数据}

\begin{itemize}
    \item[] \textbf{\href{https://gis.harvard.edu/rapidroute}{RapidRoute}}: 开源的旅行时间估算系统.

    % \item[] \textbf{\href{https://dataverse.harvard.edu/dataset.xhtml?persistentId=doi:10.7910/DVN/IAYJOC}{哈佛CGA推文人口普查数据集}}:哈佛CGA地理推文档案2.0的子集,整合了全美人口普查数据.包含2012年1月至2023年7月超过20亿条带地理标签的推文记录及用户信息和人口普查变量.

    \item[] \textbf{\href{https://gis.harvard.edu/billion-object-platform-v20}{Billion Object Platform}}: 实时处理数十亿条记录的地理空间分析平台.(\href{https://www.iq.harvard.edu/news/cga-geospatial-infrastructure-enhancement}{2024 IQSS新闻通讯报道})
    
\end{itemize}
